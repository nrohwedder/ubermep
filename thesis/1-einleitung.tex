\chapter{Einleitung}
\label{cha:einleitung}

Das Ziel dieser Bachelorarbeit ist es, eine Bibliothek zu entwickeln, welche einem Nutzer die M"oglichkeit bietet ein Overlay-Peer-to-Peer Netzwerk aufzubauen, um mit Hilfe von Nachrichtentypen eines \emph{Message Exchange Patterns} einen Informationsaustausch zu erm"oglichen.

\section{Herangehensweise}

Im Rahmen dieser Arbeit wird dabei im ersten Schritt das ben"otigte Message Exchange Pattern definiert und spezifiziert. Dabei handelt es sich um Nachrichtentypen der Kommunikationsform \emph{Unicast} und \emph{Multicast}. Im zweiten Schritt wird mittels dieser Definition ein Protokoll entworfen, welches die Kommunikation der verschiedenen Nachrichtentypen erm"oglicht. Im dritten Schritt werden Technologien gesucht, welche zum einen den Aufbau eines Peer-to-Peer Netzwerk und zum anderen den Nachrichtenaustausch mit Hilfe der Protokollschicht erm"oglichen. Im vierten und letzten Schritt wird schlie"slich eine Architektur entworfen und implementiert, welche sich an der gew"ahlten Technologie sowie der entwickelten Protokollschicht orientiert. 

\section{Gliederung der Arbeit}
Nach der oben erfolgten Einf"uhrung in die Herangehensweise, ist die Arbeit im Folgenden in 5 Kapitel aufgeteilt.

Im Kapitel \ref{cha:grundlagen} wird dem Leser ein "Uberblick "uber die Grundlagen gegeben, auf denen diese Arbeit basiert. Zun"achst wird in Abschnitt \ref{sec:p-2-p} das Konzept der Peer-to-Peer-Architektur erkl"art. Was ist die Besonderheit gegen"uber dem klassischen Client-Server-Architektur-Ansatz, was f"ur Topologien gibt es und was sind Peer-to-Peer-Overlay-Netzwerke? Danach wird das Verfahren der Remote Procedure Calls in Abschnitt \ref{sec:rpc} erl"autert. Anschliessend werden in den Abschnitten \ref{sec:netty}, \ref{sec:google-protobuf} und \ref{sec:uberlay}, das Netty-Framework, das Google-Protobuf-Projekt und das Uberlay-Projekt erkl"art, welche die Basis der entwickelten Bibliothek bilden. Abschlie"send wird in den Abschnitten \ref{sec:maven} und \ref{sec:mockito} noch kurz auf das Build-Tool Maven sowie das Test Framework Mockito eingegangen.

Das Kapitel \ref{cha:entwurf} gibt dem Leser einen "Uberblick "uber den Entwurf der Implementierung. Dazu werden in Abschnitt \ref{sec:mep} die zugrundeliegenden Message Exchange Patterns definiert und spezifiziert. Anschlie"send wird in Abschnitt \ref{sec:architektur} das Architektur-Konzept beleuchtet, um abschliessend das der Bibliothek zugrundeliegende Protokoll in Abschnitt \ref{sec:protokoll} zu definieren und zu beschreiben.

Im Kapitel \ref{cha:implementierung} wird die Implementierung des im Kapitel \ref{cha:entwurf} beschriebenen Entwurfs vorgestellt. Dazu wird zun"achst in Abschnitt \ref{sec:einleitung} eine Einleitung sowie ein kurzer "Uberblick dar"uber gegeben, wie die Implementierung aufgebaut ist. Anschlie"send wird in Abschnitt \ref{sec:impl_protokoll} die Implementierung des Protokolls beschrieben. Danach wird der Kern der Implementierung in Abschnitt \ref{sec:ubermep-core} vorgestellt. Dazu werden im einzelnen die Schnittstellen, die Implementierungen einzelner Komponenten, die Implementierung des Messaging-Systems sowie die Konfiguration einzelner Peers vorgestellt. Anschlie"send wird anhand von Beispielen in Abschnitt \ref{sec:beispiele} veranschaulicht, wie die Implementierung benutzt wird. Abschlie"send wird in Abschnitt \ref{sec:zukunft} noch kurz auf zuk"unftige Erweiterungen eingegangen.

Im Kapitel \ref{cha:evaluation} wird die Evaluation der Implementierung beschrieben. Dazu werden zun"achst die verwendeten Netzwerk-Topologien und anschlie"send die Evaluationskriterien erl"autert. Abschlie"send werden die gemessenen Ergebnisse pr"asentiert und bewertet.

Das Kapitel \ref{cha:zusammenfassung} liefert eine kurze Zusammenfassung dieser Bachelorarbeit, einen kurzen Ausblick "uber die Zukunft von diesem Projekt sowie einen "Uberblick in welchen Anwendungsbereichen die Implementierung verwendet werden kann.