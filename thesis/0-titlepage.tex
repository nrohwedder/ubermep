%---------------------------------------------------------------------------
% Frontpage
%---------------------------------------------------------------------------

\begin{titlepage}

\title{Design und Implementierung eines Peer-to-Peer-basierten Overlaynetzwerks}
\author{Nils Rohwedder}

\let\footnotesize\small
	\let\footnoterule\relax
	\null
	\vfil
	\vskip 30pt
	\begin{center}
		{\LARGE
		  {\includegraphics[width=80mm]{img/Logo_Inst_Telematik_cropped.pdf}
			\\
			\vskip 20pt
			\Large Universit"at zu L"ubeck}\\
			Institut f"ur Telematik\\[2cm]
			{\Large Bachelorarbeit}\\ [2cm]
			Design und Implementierung eines Peer-to-Peer-basierten Overlaynetzwerks
			\par}%
		\vskip 6em
		{\large \lineskip .75em
		\begin{tabular}[t]{c}
			{\Large von}\\[.5em]
			{\Large Nils Rohwedder}\\[7em]
			{\bf Aufgabenstellung und Betreuung:}\\[.5em]
			Dr.-Ing. Dennis Pfisterer \\
			Daniel Bimschas, M.Sc.
		\end{tabular}
		\par}%
		\vfill 
		{\large
			L�beck, den \today
			\par}%
	\end{center}
	\par
	% thanks
	\vfil
	\null
\end{titlepage}

\cleardoublepage

% Erklaerung
\newpage
\vspace*{7cm}
\centerline{\bf Erkl"arung}

\vspace*{1cm}
Ich versichere, die vorliegende Arbeit selbstst"andig und nur unter Benutzung
der angegebenen Hilfsmittel angefertigt zu haben.

\vspace*{3cm}
L�beck, den \today 

\pagestyle{headings}

\cleardoublepage

\section*{Aufgabenstellung}

Im Rahmen der Europ"aischen FIRE-Initiative (\glqq Future Internet Research and Experimentation\grqq, http://www.ict-fireworks.eu), die zum Gro"sbereich \glqq Future Internet\grqq \ (http://www.future-internet.eu) geh"ort, geht es um die Entwicklung von Testinfrastrukturen f"ur die Erforschung von Technologien f"ur das Internet der Zukunft. Das Institut f"ur Telematik war als Konsortialf"uhrer des inzwischen abgeschlossenen WISEBED-Projekts an FIRE beteiligt. In WISEBED wurde die Infrastruktursoftware \emph{Testbed Runtime} entwickelt, welche es erlaubt eine weltweite Testumgebung f"ur Algorithmen, Protokolle und Anwendungen f"ur drahtlose Sensornetzwerke zu betreiben. Testbed Runtime arbeitet dabei verteilt auf einer Vielzahl von Infrastrukturkomponenten der PC-, Server- oder Netbook-Klasse und verwendet zur Kommunikation zwischen den Komponenten ein selbstentwickeltes Overlay-Netzwerk.

Im Rahmen dieser Bachelorarbeit soll eine Middleware-L"osung f"ur ein solches Overlay-Netzwerk entwickelt werden. Hierbei soll die zu entwickelnde Middleware verschiedene Muster des Nachrichtenaustauschs (\emph{Message Exchange Pattern}) wie z.B. Unicast, Multicast und RPC zwischen den Peer-Hosts des Overlay-Netzwerks erlauben und realisieren. Die entwickelte L"osung soll dabei sowohl das aktuelle Overlay-Netzwerk in Testbed Runtime ersetzen, jedoch auch in anderen verteilten Anwendungen einsetzbar sein. Zu diesem Zweck sollen daher im Rahmen dieser Arbeit g"angige Message Exchange Pattern analysiert, beschrieben und implementiert werden. Es soll darauf geachtet werden ausgereifte, bereits existierende Technologien zu verwenden. Die Funktionalit"at sowie ausreichende Performanz der Implementierung soll schlie"slich durch eine Evaluation gezeigt werden.