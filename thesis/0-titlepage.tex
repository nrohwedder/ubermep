%---------------------------------------------------------------------------
% Frontpage
%---------------------------------------------------------------------------

\begin{titlepage}

\title{Latex-Template einer schriftliche Ausarbeitung}
\author{Max Mustermann}

\let\footnotesize\small
	\let\footnoterule\relax
	\null
	\vfil
	\vskip 60pt
	\begin{center}
		{\LARGE
			{\Large Universit"at zu L"ubeck}\\
			Institut f"ur Telematik\\[2cm]
			{\Large Bachelorarbeit}\\ [2cm]
			Design und Implementierung eines Peer-to-Peer-basierten Overlaynetzwerks
			\par}%
		\vskip 6em
		{\large \lineskip .75em
		\begin{tabular}[t]{c}
			{\Large von}\\[.5em]
			{\Large Nils Rohwedder}\\[7em]
			{\bf Aufgabenstellung und Betreuung:}\\[.5em]
			Dr.-Ing. Dennis Pfisterer \\
			Daniel Bimschas, M.Sc.
		\end{tabular}
		\par}%
		\vfill 
		{\large
			L�beck, den \today
			\par}%
	\end{center}
	\par
	% thanks
	\vfil
	\null
\end{titlepage}

\cleardoublepage

% Erklaerung
\newpage
\vspace*{7cm}
\centerline{\bf Erkl"arung}

\vspace*{1cm}
Ich versichere, die vorliegende Arbeit selbstst"andig und nur unter Benutzung
der angegebenen Hilfsmittel angefertigt zu haben.

\vspace*{3cm}
L�beck, den \today 

\pagestyle{headings}

\cleardoublepage

\section*{Aufgabenstellung}

%Peer-to-Peer-basierte Overlaynetzwerke sind heterogene Systeme, meist durch unterschiedliche physikalische Dom�nen getrennt.

Im Rahmen der T"atigkeit an dem WISEBED-Projekt am Institut f"ur Telematik der Universit"at zu L"ubeck tauchte die Fragestellung nach einem Peer-to-Peer basierten Overlay-Netzwerk auf, welches die M"oglichkeit eines Nachrichtenaustauschs bietet.

\begin{quote}
In WISEBED wird eine Infrastruktur entwickelt, dessen Ziel die Bereitstellung einer europaweiten Testumgebung f"ur Algorithmen, Protokolle und Anwendungen f"ur drahtlose Sensornetzwerke ist. Das Hauptaugenmerk liegt dabei zum Einen auf der Verf"ugbarkeit einer gro"sen Anzahl von Sensorknoten und zum Anderen auf der technischen Vielfalt der einzelnen vernetzten Testumgebungen, der Sensorik und der Infrastruktur zum Management und zur Nutzung dieses Sensorverbundes. Das Ziel von WISEBED ist prim"ar der Aufbau und die Bereitstellung eines gr"o"seren Sensorverbundes, um theoretisch entwickelte Algorithmen f"ur Sensornetzwerke auch praktisch in gr"o"serem Ma"sstab testen zu k"onnen.

W"ahrend der Laufzeit von WISEBED entsteht dabei ein gro"ses, europaweit verteiltes und "uber das Internet verbundenes System von Sensornetzwerken. Eine Software soll es Benutzern der WISEBED-Infrastruktur gestatten, nicht nur auf ein einzelnes Sensornetzwerk zugreifen zu k"onnen, sondern eine F"oderation aus Einzelnetzwerken zu erstellen, die aus Anwendersicht wie ein homogenes, einzelnes Sensornetzwerk erscheint und so die europaweite Verteiltheit verbirgt. Dies erlaubt die F"oderation existierender Sensornetzwerke zu einem virtuellen Gesamtgebilde in dem Nachrichten so ausgetauscht werden k"onnen, als w"aren die einzelnen Sensorknoten in gegenseitiger Funkreichweite. \cite{wisebed}
\end{quote}

Das Hauptaugenmerk dieser Arbeit liegt dabei auf einem Konzept f"ur den Nachrichtenaustausch, welches verschiedene Nachrichtentypen eines \emph{Message Exchange Pattern} abdeckt. Es soll darauf geachtet werden, dass nicht ein f"ur WISEBED spezifisches Pattern entwickelt wird. Daf"ur sind die verschiedensten Szenarien anderer Anwendungsbereiche verteilter Peer-to-Peer-Architekturen denkbar. Des weiteren soll ein effizientes \emph{Reliable Messaging-} System integriert werden, welches dem Nutzer einen schnellen und zuverl"assigen Nachrichtenaustausch erm"oglicht. Dabei soll dem Nutzer die freie Wahl des zu "ubertragenen Inhalts "uberlassen werden, sowie gegebenfalls Erweiterungspotenzial f"ur zus"atzliche Nachrichtentypen bieten. Aufbauend auf einer Analyse der Anforderungen soll hierbei zun"achst ein Konzept entworfen und anschlie"send implementiert werden. Es soll darauf geachtet werden ausgereifte, bereits existierende Technologien zu verwenden. Die Funktionalit"at sowie ausreichende Performanz der Implementierung soll schlie"slich durch eine Evaluation gezeigt werden.
