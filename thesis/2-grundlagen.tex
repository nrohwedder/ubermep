\chapter{Grundlagen}
\label{cha:grundlagen}

Im diesem Kapitel werden die technischen Grundlagen dieser Arbeit er"ortert, welche im Entwurf und der daraus resultierenden Implementierung angewendet werden. Im Abschnitt \ref{sec:p-2-p} wird das Peer-to-Peer Architektur-Konzept vorgestellt, welches die Basis dieser Arbeit bildet. Anschlie"send wird im Abschnitt \ref{sec:rpc} das Verfahren von Remote Procedure Calls erl"autert, welches einen wesentlicher Bestandteil dieser Arbeit ist. Danach wird im Abschnitt \ref{sec:netty} das Netty-Framework beschrieben, welches zusammen mit dem  Uberlay Projekt, beschrieben in Abschnitt \ref{sec:uberlay}, die Basis des Architektur-Entwurfes bildet. Im Abschnitt \ref{sec:google-protobuf} wird das Google Protobuf Projekt beschrieben, welches die Protokoll-Schicht dieser Arbeit bildet. Danach wird in Abschnitt \ref{sec:maven} das Projektverwaltungs- und Buildtool Maven vorgestellt und erl"autert in wieweit sich dieses Tool in dieser Arbeit wiederfindet. Abschlie"send wird die Java-Programmbibliothek Mockito vorgestellt, welche in dieser Arbeit zum Testen der Implementierung verwendet wurde.

\section{Peer-to-Peer Architektur}
\label{sec:p-2-p}

\cite{peer_to_peer_system_architekturen}
Die klassische Internetkommunikation basiert auf der Client-Server-Archi\-tek\-tur. Ausgew"ahlte Server (meist abh"angig von Leistungsf"ahigkeit) bieten \\ Diens\-te, Dateien, Inhalte etc. an, die Clients konsumieren diese Ressourcen. Der Vorteil hierbei: Kennen Clients einen ausgew"ahlten Server, so lassen sich Dienste, Daten oder Inhalte leicht auffinden. Allerdings besitzt diese Architektur einige Nachteile: 
\begin{itemize}
\item Ist ein Server vor"ubergehend nicht erreichbar, so sind deren s"amtlich angebotene Ressourcen ebenfalls nicht verf"ugbar. Dies entspricht dabei dem System des \emph{Single Point of Failure}, d.h. der Ausfall des Servers sorgt f"ur den Zusammenbruch des ganzen Systems (Netzwerk). Dies kann zwar durch z.B. redundante Datenhaltung verhindert werden, verschiebt dabei aber nur das eigentliche Problem des Systems und l"ost es nicht.
\item Die theoretisch zur Verf"ugung stehenden Ressourcen von Clients sind f"ur andere Clients nicht verf"ugbar. So ist z.B. das \emph{Filesharing}, bei dem Nutzer untereinander Dateien teilen oder das Nutzen der \emph{Grid Computing} Technologie in dem Rechenleistung geteilt wird, nicht oder nur bedingt m"oglich.
\end{itemize}

\subsection{Was ist Peer-to-Peer?}

Die Peer-to-Peer-Architektur ist ein alternatives Konzept um diese Nachteile aufzul"osen.

\myfigtwo[cs_p2p_arch]{Client-Server Architektur}{Peer-to-Peer Architektur}{Client-Server vs. Peer-to-Peer}

Peer-to-Peer-Netzwerke sind verteilte Systeme, bestehend aus mehreren Knoten, die {\it Peers} genannt werden. Jeder Peer im Netzwerk ist ein gleichwertiger Partner ({\it Peer:} engl. f"ur \emph{Gleichgestellter}) und kann sowohl Client- als auch Server-Aufgaben ausf"uhren. Peers eines Netzwerks kommunizieren unmittelbar miteinander, (siehe Abbildung \ref{fig:cs_p2p_arch}) dies bedeutet aber \emph{nicht}, dass jeder Peer mit jedem weiteren Peer des Netzwerks verbunden sein mu"s. Es bedeutet lediglich, dass Peers keine Server ben"otigen, um Informationen auszutauschen. So ist im Beispiel in Abbildung \ref{fig:cs_p2p_arch} einzig Peer 1 mit Peer 2, Peer 2 mit Peer 3 und Peer 4 und Peer 3 mit Peer 4 verbunden. Das Peer-to-Peer-Architektur-Konzept erlaubt aber trotzdem die Kommunikation zwischen s"amtlichen Peers des Netzwerk (wie z.B. zwischen Peer 1 und Peer 4).

\subsection{Routing}
\label{subsec:p2p-routing}
Um die Kommunikation zwischen Peers eines Netzwerks, unabh"angig ihrer Topologie, zu erm"oglichen, m"ussen die Informationen, wo und wie einzelne Peers zu erreichen sind, f"ur jeden Peer verf"ugbar sein; (m"ochte, bezogen auf Abbildung \ref{fig:cs_p2p_arch-b}, z.B. Peer 1 mit Peer 4 kommunizieren). Diese Informationen wo und wie Peers erreicht werden k"onnen, werden in sogenannten {\it Routing-Tabellen} gespeichert. Die Such- und Weiterleitungsverfahren, welche die Routing-Tabellen zum Auffinden von Knoten in Netzwerken ben"otigen, nennt man dann {\it Routingverfahren}. Das Routingverfahren eines Netzwerks ist implementierungsabh"angig, das bedeutet, jedes Peer-to-Peer Netzwerk kann sein eigenes Routingverfahren besitzen. Bekannte Routingverfahren in Peer-to-Peer-Netzwerken sind z.B. \emph{CAN (Content Addressable Network)} \cite{dht_can} oder \emph{Chord} \cite{dht_can} welche auf der Datenstruktur \emph{DHT (Distributed Hash Table)} \cite{dht_can} aufsetzen.

\nomenclature{DHT}{Distributed Hash Table}
\nomenclature{CAN}{Content Addressable Network}

%{\it Distance Vector Routing} oder {\it Link State Routing}. Ein m"ogliches Routingverfahren, welches zur Grupper der {\it Distance Vector Routing}-Verfahren geh"ort, wird dabei %im Abschnitt \ref{subsec:uberlay-pvp} erkl"art. 

Ein Peer-to-Peer Netzwerk ist hoch-dynamisch. Da Knoten sich jederzeit vom Netzwerk an und wieder abmelden k"onnen oder Knoten vor"ubergehend nicht verf"ugbar sein k"onnen, m"ussen sich die Routing-Tabellen dynamisch an das Netzwerk anpassen. D.h. Peer-to-Peer Netzwerke organisieren sich selbst. Daraus resultiert eine hohe Fehlertoleranz, da selbst durch den Ausfall von einem oderer mehrerer Knoten die Kommunikation weiterhin gew"ahrleistet bleibt. 

%Des weiteren besitzen die Kommunikationsverbindungen (Routes) in Peer-2-Peer-Netzwerken metrische Daten (Metrik), welche die G"ute eine Verbindung beschreiben (z.B. %Berechnung des Zeitabschnitts vom Versenden einer Nachricht an einen Peer bis zum Empfangen seiner Antwort). 

\subsection{Peer-to-Peer Overlay-Netzwerk}
Die Kommunikation eines Peer-to-Peer Overlay-Netzwerks findet auf einer eigenen Netzwerktopologie, dem Overlay, statt. Diese Topologie setzt auf eine bestehende physikalische Netzwerktopologie, dem Underlay, auf. Der Vorteil hierbei ist, dass das Routen und Addressieren von Peers unabh"angig von der pysikalischen Netzwerkstruktur arbeitet. Aus diesem Grund besitzen Overlay-Netzwerke typischerweise eigene Addressierungen und Routing-Verfahren.

\myfig{cs_p2p_overlay}{Netzwerk mit einer Overlay- und einer Underlay-Topologie}

Ein Peer-to-Peer-Overlay-Netzwerk mit einer darunterliegenden Underlay Topologie ist in Abbildung \ref{fig:cs_p2p_overlay} anhand eines Beispiels zu erkennen. Die physikalische Netzwerktopologie (Underlay) besteht dabei aus 6 Hosts, wobei jeweils 3 Hosts mittels einem Router "uber das Internet miteinander verbunden sind. Das Overlay wiederum besteht aus 6 Peers. Jedem Peer wird dabei jeweils ein Host des Underlay zugeordnet. Die Peers sind ringf"ormig miteinander verbunden und besitzen eigene Addressierungs- sowie Routingverfahren f"ur Nachrichten, die  unabh"angig von der Struktur des Underlay arbeiten.

\subsection{Topologien}
\label{subsec:p2p-topo}
Es wird zwischen zwei Arten von Peer-to-Peer-Overlay-Netzwerk-Topologien unterschieden: {\it Single-Hop} und {\it Multi-Hop}. Ein {\it Hop} entspricht dabei einer Etappe von einem Netzknoten zum n"achsten. Im Folgenden wird nun der Unterschied dieser beiden Modelle erkl"art.

\paragraph{Single-Hop}
Single-Hop-Topologie bedeutet: Bei der Kommunikation von einem Peer des Netzwerks zu einem beliebig anderen Peer wird {\it immer nur ein} Hop, also eine Etappe zur"uckgelegt.

\paragraph{Multi-Hop}
Multi-Hop-Topologie bedeutet: Bei der Kommunikation von einem Peer des Netzwerks zu einem beliebig anderen Peer, wird {\it immer mindestens ein} Hop zur"uckgelegt. Das bedeutet, ein Gro"steil der Daten die "uber\-tra\-gen werden, laufen "uber Peers die als Zwischenstationen fungieren. So ist z.B. die in Abbildung \ref{fig:cs_p2p_overlay} beschriebene Overlay-Topologie eine Multi-Hop-Topologie, da der Peer 1 eine Nachricht zum Peer 3 "uber den Peer 2 versenden muss.



\section{Remote Procedure Call (RPC)}
\label{sec:rpc}

\nomenclature{RPC}{Remote Procedure Call}

Remote Procedure Call (RPC) ist ein Verfahren zur Abwicklung von Interprozesskommunikation welches den Austausch von Daten in verteilten Systemen erlaubt. 
\myfig[10 cm]{rpc_arch-a}{"Ubertragung eines Remote-Procedure-Call}
Daf"ur muss der Aufruf, der von einem Client zum aufrufenden Server "uber\-tra\-gen wird, in eine Nachricht serialisiert werden. Diese Serialisierung nennt man {\it Marshalling}. Der serialisierte Aufruf wird dann "uber einen Kommunikationskanal "uber\-tra\-gen und beim Server deserialisiert, die Deserialisierung hei"st {\it Unmarshalling}. Der Ablauf ist dabei in Abbildung \ref{fig:rpc_arch-a} zu erkennen.

Dieses Verfahren erlaubt den Aufruf von entfernten Prozeduren (z.B. auf einem verbundenen Remote-Peer). Der Konsument wird dabei im Allgemeinen als Client, der Dienstleister als Server bezeichnet.

\subsection{Ablauf}
\myfig[13 cm]{rpc_arch-b}{Ablauf eines Remote-Procedure-Call}
 Der Ablauf eines Remote Procedure Calls sieht wie folgt aus:

Auf dem Client wird f"ur eine Applikationslogik (1) mittels einem Service-Inter\-face (2) ein Stub (Stellvertreter) generiert. Dieser Stub serialisiert den Aufruf (Marshalling), welcher dann "uber einen Kommunikationskanal an den Server "uber\-tra\-gen wird (3). Auf dem Server wird dann die serialisierte Nachricht vom Skeleton (4) in einen Aufruf deserialisiert (Unmarshalling). Mittels des Service-Interface (5) (dasselbe wie am Client) wird dann die entsprechende Service-Implementierung aufgerufen (6). Das Ergebnis wird "uber den Server-Skeleton serialisiert, an den Stub des Clients "uber\-tra\-gen und anschliessend wieder deserialisiert. Das Ergebnis kann schlie"slich in der Applikationslogik am Client verarbeitet werden.

Die Abbildung \ref{fig:rpc_arch-b} zeigt dabei den oben beschriebenen Ablauf eines RPC.

\subsection{Realisierungen}
F"ur die Verwendung von RPC gibt es mehrere verschiedene Arten der Realisierung. Im Zusammenhang anwendungssorientierter Middleware w"are da zum einen \emph{CORBA} \cite{corba} zu nennen, eine Spezifikation f"ur objektorientierte Programmiersprachen. Im Bereich kommunikationsorientierter Middleware w"are da \emph{Web services} \cite{web_services} zu nennen, ein System um den Aufruf verteilter Webanwendungen zu unterst"utzen. In dieser Implementierung wird die vom Google-Protobuf-Projekt \cite{protobuf} mitgelieferte RPC-Variante realisiert (siehe dazu Abschnitt \ref{sec:google-protobuf}).

\section{Netty}
\label{sec:netty}
	
Die Implementierung dieser Arbeit basiert auf Netty \cite{netty}, einem asynchronen \emph{network application framework}.
Die Architektur setzt sich dabei aus drei Teilen zusammen:
\begin{itemize}
\item {\bf Pipelines:} Der Transport in Netty findet "uber Sockets statt. Jeder Socket ist mit einer \emph{ChannelPipeline} verbunden, welche \emph{ChannelHandler} bereith"alt, die f"ur die Verarbeitung der Nachrichten sorgen. Die Nachrichten-Pakete werden dabei mittels einem \emph{Event-Driven} Client-Server-Model verarbeitet.
\item {\bf Codecs:} Die zu "ubermittelnden Objekte werden mittels entsprechender Decoder und Encoder in Nachrichten-Pakete umgewandelt.
\item {\bf Buffers:} Die Nachrichten-Pakete werden mittels des \emph{Zero-Copy Mechanismus} in \emph{ChannelBuffer} verpackt und dann versendet.
\end{itemize}

Im Folgenden werden zun"achst die beiden eingef"uhrten Begriffe, Event-Driven Architektur und Zero-Copy Mechanismus erkl"art, bevor anschliessend die Architektur selbst beschrieben wird.

\subsubsection{Event-Driven Architektur}
Normalerweise wird auf Multi-Thread-Servern beim Empfangen von Requests ein Thread pro Request erzeugt. Dies birgt mehrere Nachteile: 
\begin{itemize}
\item Da der Thread die meiste Zeit mit dem Warten auf I/O-Operationen verbringt, befindet sich der Thread den gro"steil seiner Zeit im Zustand des Leerlaufs. Daraus ergibt sich eine Verschwendung von Ressourcen.
\item Au"serdem ist das Erzeugen von Threads teuer und bringt den Seiteneffekt von Verz"ogerungszeiten mit sich. Das Verwenden von Threadpools kann dies zwar verbessern, aber nicht verhindern.
\end{itemize}

Genau hier setzt die \emph{Event-Driven Architektur} an. Jeder Aufruf einer I/O-Operation f"uhrt zum ausl"osen eines nicht-blockierenden Events. So wird z.B. ein Event ausgel"ost, wenn Daten auf ein Socket geschrieben werden. Das Verbinden bzw.Trennen von Kommunikationskan"alen f"uhrt ebenfalls zum ausl"osen eines Events. 
Dabei wird das Verarbeiten eines Events an einen sogenannten {\it Worker Thread}, aus einem Pool von Threads, weitergeleitet, welcher dann das Event an die Handler in der Pipeline weiterreicht. Dabei k"onnen sich verschiedene Events einen Worker Thread teilen.

\subsubsection{Zero-Copy Mechanismus}
Beim Versenden von Nachrichten-Paketen liegen die zu versendenden Daten typischerweise auf der Festplatte oder im Kernel-eigenen Cache vor. Im Folgenden soll davon ausgegangen werden, das die Daten im Kernel-space vorliegen. 

F"ur das Schreiben der Daten auf einem Client-Socket, m"ussen diese allerdings im Socket-Buffer vorliegen. Normalerweise w"urde dies "uber einen Kopiervorgang geschehen, welcher die Daten aus dem Kernel-space "uber den Userspace in den Socket-Buffer kopiert. Allerdings hat der Kopiervorgang vom Kernel-space in den User-space den Nachteil der schlechten Performance, zumal sich der Inhalt nicht ver"andert hat.

Anstatt diese Daten nun hin und her zu kopieren, wird beim Zero-Copy-Ansatz ein virtueller Puffer im User-space erzeugt, welcher einzig {\it Pointer} auf den tats"achlichen Inhalt im Kernel-space generiert. Wird dann beim Schreiben der Daten in den Socket-Buffer auf den Inhalt zugegriffen, wird lediglich der Pointer gelesen und der Speicherinhalt vom Kernel-space via {\it Direct Memory Access} in den Socket-Buffer geschrieben. Derselbe Mechanismus greift dann ebenso beim Empfangen von Nachrichten am Server, also beim Verschieben von Daten-Paketen aus dem Socket-Buffer in den Kernel-space.

Eine ausf"uhrlichere Beschreibung der beiden Mechanismen -- Event-Driven und Zero-Copy Mechanismus -- ist auf der Dokumentationsseite von Netty unter \url{http://www.jboss.org/netty/documentation.html} zu finden.

In dieser Arbeit spielt der Zero-Copy Mechanismus f"ur Nachrichten von geringer Gr"o"se eine ungeordnete Rolle, gewinnt aber an Bedeutung, je gr"o"ser die zu versendenden Nachrichten werden. 

\subsection{Architektur}
\label{subsec:netty-arch}
\myfig{netty-arch}{Vereinfachte Netty-Pipeline-Architektur}
Im folgenden soll nun auf die wesentlichen, f"ur diese Arbeit relevanten Aspekte der Netty-Architektur eingegangen werden. Hierf"ur soll die Abbildung \ref{fig:netty-arch} die Architektur des Netty-Frameworks veranschaulichen. Sie spiegelt dabei aber nicht die tats"achliche Realisierung in Netty wieder. 

F"ur weiterf"uhrende Information, die "uber diese Arbeit hinaus gehen, steht die Netty-Projektseite zur Verf"ugung. (\url{http://www.jboss.org/netty})

\subsubsection{Pipelines und Buffers}
Das Netty-Framework bietet die M"oglichkeit Nachrichten von einem Client-Socket zu einem Server-Socket zu senden. Jeder Socket besitzt eine {\bf ChannelPipeline}.
Die Kommunikation findet dabei "uber einen {\bf Channel} statt, welcher zwischen einem Client-Socket und einem Server-Socket aufgebaut wird. Wie bereits erw"ahnt, basiert Netty auf der Event-driven Architektur, d.h. die Nachrichten werden als {\bf ChannelEvents} verschickt. Dabei gibt es verschiedene Arten von ChannelEvents. Am wichtigsten hierbei ist das {\bf MessageEvent}, welches den Empf"anger eines Events, den {\bf ChannelFuture} (also das Ergebnis der asynchronen Verarbeitung) sowie die Nachricht selbst enth"alt. 

Wie in Abbildung \ref{fig:netty-arch} zu erkennen ist, enth"alt die ChannelPipeline beliebig viele {\bf ChannelHandler}, die dann f"ur die Verarbeitung der Nachricht sorgen. Das Prinzip hierbei ist wie folgt: Die Nachrichten werden in der Pipeline von einem Handler zum n"achsten weitergereicht. F"ur die Richtung ist dabei entscheidend, ob eine Nachricht downstream (sendend) oder upstream (empfangend) verarbeitet wird. Downstream werden die Nachrichten von oben nach unten (top-down) in der Pipeline weitergereicht, upstream von unten nach oben (bottom-up). Ist ein Handler f"ur eine Nachricht verantwortlich, so wird diese dort verarbeitet und anschlie"send weitergereicht, wenn nicht wird sie direkt, ohne Verarbeitung, weitergereicht. ChannelHandler k"onnen dabei unterschiedliche Aufgaben erf"ullen, wie z.B. das De-/Encodieren von Nachrichten oder das Abarbeiten einer Applikationslogik.
Die ChannelHandler in der Pipeline sind positionsbezogen. Beim Versenden muss schlu"sendlich der letzte, f"ur den Nachrichtentyp verantwortliche ChannelHandler in der ChannelPipeline, daf"ur sorgen, dass die Nachricht serialisierbar ist, d.h. vom Framework "uber den Channel versendet werden kann.

Es gibt dabei zwei Arten von ChannelHandlern: Klassen die das Interface {\bf ChannelDownstreamHandler} implementieren sind f"ur das Verarbeiten von versendeten Nachrichten und Klassen die das Interface {\bf ChannelUpstreamHandler} implementieren f"ur das Verarbeiten von empfangenden Nachrichten zust"andig.

Die Nachricht selbst, wird in einen {\bf ChannelBuffer} verpackt und dann "uber\-tra\-gen. Dabei wendet Netty f"ur den Kopiervorgang den bereits beschriebenen Zero-Copy Mechanismus an. Der ChannelBuffer selbst besteht aus einem zusammengesetzten byte-array und h"alt f"ur die primitiven Datentypen eigene Methoden wie, \emph{readByte}, \emph{readInt}, \emph{writeByte}, \emph{writeInt}, ... bereit.

\subsubsection{Codecs}
Die Serialisierung bzw. Deserialisierung von Nachrichten findet bei Netty "uber Encoder (Serialisierer) bzw. Decoder (Deserialisierer) statt.
Die Dekodierte Nachricht, also die eigentliche Protokoll-Nachricht, wird dann "uber die Channels mittels der ChannelEvents versendet.

F"ur das De- bzw. Enkodieren von Nachrichten gibt es zwei M"oglichkeiten: Zum Einen das Verwenden von Netty bereitgestellten Dekodern und Enkodern (wie z.B. {\bf HttpRequestEncoder}, {\bf HttpRequestDecoder} und {\bf HttpResponseEncoder}, {\bf HttpResponseDecoder}), zum Anderen das Implementieren von eigenen De- bzw. Enkodern. Die Implementierung von eigenen De- bzw. Enkodern soll hier nicht n"aher beschrieben werden, zumal dies anhand von Beispiel-Applikationen auf der Netty-Homepage nachvollzogen werden kann. Die genutzten De- und Enkoder m"ussen dann in der ChannelPipeline hinzugef"ugt werden. 

In dieser Implementierung werden die Protokoll-Nachrichten unter Verwendung des Google-Protobuf-Projekts \cite{protobuf} versendet. Um diese nutzen zu k"onnen, m"ussen als De- bzw. Enkoder die von Netty mitgelieferten {\bf ProtobufDecoder} und {\bf ProtobufEncoder} der Pipeline hinzugef"ugt werden.

Im folgenden wird nun auf die Google-Protobuf-Nachrichten eingegangen.

\section{Google Protocol Buffers}
\label{sec:google-protobuf}

\nomenclature{IDL}{Interface Definition Language}

Google Protocol Buffers (protobuf) ist ein Format zur Serialisierung von strukturierten Daten (wie z.B. Nachrichtenprotokollen) mittels einer {\it Interface Definition Language} (IDL). 
Google Protobuf erlaubt dabei eine programmiersprachenunabh"angige und effiziente Protokollnachrichten-Serialisierung. 

Die Effizienz wird unter anderem dadurch erreicht, dass f"ur die Serialisierung von zu "uber\-tra\-genen Daten eine Kodierung variabler L"ange, sogenannte \emph{Varints}, verwendet werden. Dabei gilt die Regel: Je gr"o"ser die Zahl, desto mehr Bytes werden ben"otigt, d.h. die Anzahl der Bytes ist abh"angig von der Gr"o"se der Zahl. Dabei wird diese Form der Serialiserung eben nicht nur f"ur Zahlen (wie z.B. {\it int32}, {\it int64}), sondern auch f"ur andere Datentypen (wie z.B. f"ur L"angenfelder von {\it Strings}) verwendet \cite{protobuf_encoding}. Eine genaue Erkl"arung wie Varints enkodiert bzw. dekodiert werden, findet sich auf der entsprechenden Dokumentationsseite von protobuf \cite{protobuf_encoding}. 

Des weiteren wurde protobuf als Bin"arformat konstruiert und ist deshalb schneller als andere Formate die nicht Bin"ardaten enthalten, aber dasselbe Prinzip verfolgen, wie z.B. textbasierte Serialisierungsformate wie XML.

Um nun strukturierte Daten zu serialisieren, mu"s man diese mittels einer \emph{.proto}-Datei beschreiben. Diese {\it proto}-Dateien werden dann mit Hilfe des Protobuf-eigenen IDL-Compilers \emph{protoc} generiert. Dies bietet den Vorteil der einfachen Erweiterbar- bzw. Austauschbarkeit von Protokoll-Nachrichten.

Die Erweiterbarkeit bzw. Austauschbarkeit ist dadurch gegeben, dass zus"atzlich ben"otigte Felder in der proto-Datei hinzugef"ugt bzw. ausgetauscht werden k"onnen und die Datei dann nur noch erneut kompiliert werden mu"s.

\subsection{Nachrichten-Generierung}

Im folgenden (Listing \ref{lst:protobuf}) soll eine Beispiel-{\it Proto}-Nachricht zum Versenden von Buchbestellungs-Nachrichten den Aufbau von Google Protobuf-Nachrichten veranschaulichen.
Die Kommunikation erfolgt dabei mittels einem \emph{BookOrderRequest}- und \emph{BookOrderResponse}-Nachrichtentyp. Die Nachrichtentypen bestehen dabei aus erforderlichen (\emph{required}), optionalen (optional) und mehrfachen (repeated) Attributen, die wiederum einen Typ, einen Bezeichner und eine Reihenfolge besitzen. Die Typen bestehen dabei aus von protobuf mitgelieferten Datentypen, aus enums sowie aus den eigenen definierten Nachrichtentypen \emph{Customer} und \emph{Book}.

\begin{lstlisting}[label=lst:protobuf, caption=Aufbau einer proto-Nachricht ,tabsize=2]
message BookOrderRequest{
	required Customer customer = 1;
	required Book book = 2;
	required bool deliverToCustomer = 3;
}

message BookOrderResponse{
	required uint64 orderNumber = 1;
}

message Customer{
	required uint32 customerNumber = 1;
	enum Gender{
		UNKNOWN = 0;
		MALE = 1;
		FEMALE = 2;
	}
	required Gender gender = 2 [default = UNKNOWN];
	required string name = 3;
	required string address = 4;
	optional uint32 phone = 5;
}

message Book{
	required string title = 1;
	repeated string authors = 2;
}

\end{lstlisting}

\subsection{Protobuf und RPC}
Des weiteren bietet protobuf die M"oglichkeit RPC-Services (siehe dazu Abschnitt \ref{sec:rpc}) zu generieren, wie im folgenden Beispiel (Listing \ref{lst:protobuf-rpc}) erl"autert. 
Daf"ur wird im ersten Schritt die Option \emph{java\_generic\_services} (f"ur Java-Appli\-ka\-tionen) gesetzt, --- weitere Programmiersprachen werden auch unterst"utzt, siehe dazu \cite{protobuf_options}. Anschlie"send folgt die Beschreibung des Service, der in diesem Beispiel einen RPC-Aufruf \emph{order} erzeugt, welcher einen BookOrderRequest "ubergeben bekommt und einen BookOrderResponse zur"uckliefert.

\begin{lstlisting}[label=lst:protobuf-rpc, caption=Erweiterung der {\it proto}-Nachricht f"ur RPC, firstnumber=27]
option java_generic_services = true;

service BookOrderService {
	rpc order(BookOrderRequest) returns(BookOrderResponse);
}
\end{lstlisting}

Dieser Service muss dann nur noch der proto-Datei angeh"angt werden. 

\section{Uberlay}
\label{sec:uberlay}

Das Uberlay-Projekt \cite{uberlay} ist ein \emph{high-performance small-scale overlay network}.
Entwickelt wurde das Uberlay-Projekt am Institut der Telematik an der Universit"at zu L"ubeck. Es basiert auf dem Netty-Framework (siehe Abschnitt \ref{sec:netty}) sowie dem Google-Protobuf-Projekt (siehe Abschnitt \ref{sec:google-protobuf}). 

Uberlay unterst"utzt den Aufbau von Multi-Hop Netzwerken (siehe Abschnitt \ref{subsec:p2p-topo}). Da in dieser Arbeit Nachrichtentypen des Message Exchange Pattern versendet werden, musste das Uberlay dahingehend angepasst werden, dass die Nachrichtentypen, der verwendeten Multi-Hop Topologie, entsprechend "ubermittelt werden. Dabei wurde die Uberlay Architektur, wie in Abbildung \ref{fig:uberlay_arch} beschrieben, nicht ver"andert.

\subsection{Architektur}
\myfig{uberlay_arch}{Uberlay Layer Architektur}

\nomenclature{UP}{Uberlay Protocol}

Das Uberlay-Netzwerk lehnt sich konzeptuell stark an IP an. Hierzu wurde f"ur den Paket-Transport ein einfaches (mittels Google Protobuf erstelltes) Protokoll entwickelt: das Uberlay-Protokoll (UP).

Anwendungen, welche das Uberlay-Netzwerk verwenden, h"angen sich als ChannelHandler in die daf"ur bestimmte ChannelPipeline ({\bf Application\-Channel\-Pipeline}) ein (siehe Abbildung \ref{fig:uberlay_arch}). Dabei entspricht die ApplicationChannelPipeline einer Netty-ChannelPipeline (siehe Abschnitt \ref{sec:netty}) zur Verarbeitung von Nachrichten durch entsprechende Handler. Nachrichten werden, "ahnlich wie bei UDP, an andere Uberlay-Hosts gesendet, indem in den versendeten Protokoll-Nachrichten zus"atzlich die Adresse des Ziel-Knotens im Overlay-Netzwerk angegeben werden.

Die Klasse {\bf UberlayNexus} verbindet nun die ChannelPipeline der Applikation, mit einer Menge von Verbindungen zum Overlay-Netzwerk. Sie implementiert eine Router-Funktionalit"at zum Empfangen ({\bf UPRouter}) und Versenden von Nachrichten ({\bf ApplicationChannelSink}) f"ur UP und w"ahlt je nach Zieladdresse den n"achsten Channel zum n"achsten Hop des Multi-Hop Netzwerk. Der zu benutzende Channel wird mit Hilfe einer Routing-Tabelle ermittelt. Die Routing-Tabelle wird dabei mittels eines implementierten Pfadvektorprotokolls (siehe Abschnitt \ref{subsec:uberlay-pvp}) erzeugt bzw. aktualisiert. Die Metrik der Routing-Tabellen wiederum wird durch das Round Trip Time Protokoll (siehe Abschnitt \ref{subsec:uberlay-rtt}) bestimmt.

Die von Uberlay konfigurierten Encoder und Decoder, in der Channel\-Pipe\-line von Uberlay ({\bf UberlayChannelPipeline}), sorgen dann f"ur die Serialisierung bzw. Deserialisierung der Protokollnachrichten. Sollte die Zieladdresse der eigene Netzknoten sein, so wird die Nachricht vom UPRouter wieder an die ChannelPipeline der Applikation hoch gesendet (Loopback-Funktionalit"at).

\subsection{Pfadvektor Protokoll}
\label{subsec:uberlay-pvp}
Wie bereits in Abschnitt \ref{subsec:p2p-routing} erw"ahnt, besitzen Overlay-Netzwerke, zur Adressierung von Peers, Routingverfahren. Das {\it Pfadvektor Routing}, welches in Uberlay verwendet wird, wird nun im Folgenden erl"autert, wobei dieses Routingverfahren kein Overlay-spezifisches Verfahren ist.

\subsubsection{Algorithmus}
Das Uberlay-Peer-to-Peer Netzwerk unterst"utzt die Nutzung von Multi-Hop-Topologien (siehe Abschnitt \ref{subsec:p2p-topo}). Das bedeutet, sind Peers nicht direkt mit\-ei\-nan\-der verbunden, so m"ussen Nachrichten "uber dazwischenliegende Peers "uber\-tra\-gen werden. Die Entscheidung welche Peers f"ur die Weiterleitungen verwendet werden sollen, trifft das implementierte Routing-Protokoll.
Das hier verwendete Pfadvektor Protokoll ist eines von diesen und geh"ort zur Familie der {\it Distanzvektor Routing Protokolle}. 

 {\it Distanzvektor Protokolle} dienen dabei zur Berechnung des k"urzesten Weges von einem Start- zu einem Zielknoten. Der Algorithmus sorgt daf"ur, das in periodischen Abst"anden eine Kopie der eigenen Routing-Tabelle an die Nachbarknoten weitergegeben wird, um mittels dieser Tabellen den k"urzesten Weg zu berechnen bzw. zu aktualisieren.

Der {\it Pfadvektor Protokoll Algorithmus} arbeitet wie folgt:

Jeder Knoten speichert in seiner Routing-Tabelle nicht nur die von ihm erreichbaren Zielknoten, sondern zus"atzlich die m"oglichen Pfade zu diesem Knoten. Die so entstehende  Routing-Tabelle entspricht dann dem {\it Spannbaum} des Netzwerks. Da sich aber sowohl der ben"otigte Speicherplatz als auch die ben"otigte Suchzeit im Vergleich zu einem Spannbaum deutlich vergr"o"sert, wird h"aufig bei Implementierungen, und so auch bei Uberlay, der \emph{minimale Spannbaum} verwendet. Eventuell ausfallende Pfade werden mittels einem Aktualisierungsintervall ersetzt, wobei dieser an die zu erwartende Stabilit"at des Netzwerks angepasst werden kann. Dies ist, bezogen auf die Zeit- und Platzkomplexit"at, im Allgemeinen kosteng"unstiger als die Verwendung der gesamten Spannb"aume. So wird z.B. in Abbildung \ref{fig:uberlay_pvp-b} f"ur den Zielknoten D einzig der Pfad $A \leftrightarrow C \leftrightarrow D$ in der Routing-Tabelle gespeichert und eben nicht der auch m"ogliche Pfad $ A \leftrightarrow B \leftrightarrow D$.

Anhand dieses Pfades kann nun der g"unstigste Weg von einem Ziel- zu einem Endknoten in einem Multi-Hop-Netzwerk gefunden werden. 

\myfigtwo[uberlay_pvp]{Pfadvektor-Kostenberechnung}{RoutingTabelle von Knoten A}{Pfadvektor Protokoll-Beispiel}

\paragraph{Beispiel}
Mittels der in Abbildung \ref{fig:uberlay_pvp-b} dargestellten Pathvector-Routing-Tabelle kann nun z.B. folgende Frage beantwortet werden:
\begin{itemize}
\item Frage von E: Wie komme ich am schnellsten von A nach D?
\item Antwort von A:  "Uber die Route A $\leftrightarrow$ C $\leftrightarrow$ D, da: \\
$3 + 5 = \underline{8 < 9} = 2 + 7$
\end{itemize}

\subsection{Round Trip Time Protokoll}
\label{subsec:uberlay-rtt}
Das Round Trip Time Protokoll in Uberlay berechnet und aktualisiert gegebenfalls die Kosten (Metrik) der Routing-Tabellen-Eintr"age. Die Metrik entspricht dabei der Zeitspanne die vergeht, bis ein Datenpaket von einer Quelle zu einem Ziel und wieder zur"uck zur Quelle gereist ist. In Abbildung \ref{fig:uberlay_rtt} ist die Berechnung einer Round Trip Time (RTT) der Route A $\leftrightarrow$ B zu erkennen. 

\myfig[8 cm]{uberlay_rtt}{Berechnung einer Round Trip Time}

\section{Maven}
\label{sec:maven}
Die Entwickler des Maven-Projekts \cite{maven} sehen Maven als \emph{software project management and comprehension tool}. Maven wurde entwickelt, um den Build-Prozess eines Projekts zu vereinfachen, weshalb Maven prim"ar auch heute meist als Build-Werkzeug verwendet wird. Maven bietet Entwicklern von Projekten inzwischen aber auch eine Vielzahl von weiteren M"oglichkeiten:
\begin{itemize}
\item Erzeugen eines Build-Systems, mit dessen Hilfe ganze Projekte gebaut oder einzelne Build-Phasen durchgef"uhrt werden k"onnen, wie z.B. das Validieren auf Korrektheit der Struktur des Projektverzeichnis und Verf"ugbarkeit aller ben"otigten Projekt-Informationen oder das Testen aller integrierten Software-Tests.
\item Deklaratives Erzeugen und Verwalten von relevanten Projektinformationen wie z.B. das Verwalten von Abh"angigkeiten zu anderen Projekten.
\item Versionierung von Projekt-Releases, einschlie"slich der automatisierten Ver\-"offentlichung auf der Projektseite.
\end{itemize}
\subsection{Funktionsweise}
\subsubsection{POM}

\nomenclature{POM}{Project Object Model}

Das Grundger"ust von Maven ist das Konzept des {\bf P}roject {\bf O}bject {\bf M}odel, basierend auf einer Projektbeschreibung mittels XML in der Datei {\bf pom.xml} im Wurzelverzeichnis des Projekts. S"amtliche f"ur das Projekt relevanten Informationen werden in dieser POM, bzw. in anderen Dateien die auf diese POM verweisen, gehalten. Den Aufbau einer POM ist im Listing \ref{lst:maven} anhand eines leicht angepassten Ausschnittes des Core-Moduls der Implementierung dieser Arbeit zuerkennen.

\lstset{language=Python}
\begin{lstlisting}[label=lst:maven, caption=Aufbau einer POM]
<?xml version="1.0" encoding="UTF-8"?>
<project xmlns="http://maven.apache.org/POM/4.0.0"
         xmlns:xsi="http://www.w3.org/2001/XMLSchema-instance"
         xsi:schemaLocation="http://maven.apache.org/POM/4.0.0 http://maven.apache.org/xsd/maven-4.0.0.xsd">

	<modelVersion>4.0.0</modelVersion>

	<groupId>de.uniluebeck.itm.ubermep</groupId>
	<artifactId>core</artifactId>
	<name>Ubermep :: Core</name>
	<packaging>jar</packaging>
	<version>0.1</version>

	<dependencies>
		<dependency>
			<groupId>de.uniluebeck.itm.uberlay</groupId>
			<artifactId>core</artifactId>
			<version>0.1</version>
		</dependency>
		
		<dependency>
			<groupId>commons-logging</groupId>
			<artifactId>commons-logging</artifactId>
			<type>jar</type>
			<scope>compile</scope>
		</dependency>
	</dependencies>
	
</project>
\end{lstlisting}

Dabei haben die XML-Tags die folgende Bedeutung:

\begin{itemize}
\item {\bf groupId} enth"alt die eindeutige Produktgruppe unter welcher das Projekt entwickelt wird. (meist in der Form einer umgekehrten Domain wie: de.uniluebeck.
\item {\bf artifactId} enth"alt die eindeutige Kennung (ID) des Artefakts in der Produktgruppe. Artefakte dienen dabei dem Bauen von Projekten, z.B. beim Erzeugen einer JAR, ist der Dateiname aus \\ \texttt{$<$articfactId$>$-$<$version$>$.jar} zusammengesetzt. 
\item {\bf name} beschreibt den vollen ausgeschriebenen Namen des Projekts.
\item {\bf packaging} definiert den Erstellungstyp des Artefakts des Projekts, wie z.B. {\bf jar}, {\bf war}, {\bf ear} oder {\bf pom}. Um das Generieren k"ummern sich Maven-Plugins.
\item {\bf version} enth"alt die aktuelle Versionsnummer des Artefakts, welche zur Versionierung von Projekten ben"otigt werden.
\item {\bf dependencies} enthalten die Abh"angigkeiten auf Artefakte anderer Projekte. 
\end{itemize}

Die komplette XML-Schema-Datei f"ur Maven 4.0.0 ist unter \url{http://maven.apache.org/xsd/maven-4.0.0.xsd} verf"ugbar.

\subsubsection{Verzeichnisstruktur}
Maven erzeugt die Verzeichnisstruktur eines Softwareprojekts entsprechend des Prinzips \emph{Convention over Configuration}, d.h. es verwendet Konventionen. In einem "ubergeordneten POM, dem Super POM, werden die Konventionen definiert. Alle Untermodule erben die Eigenschaften des Super POMs. Die Untermodule, die selbst jeweils ein POM enthalten, k"onnen geerbte Attribute "uberschreiben, Ausnahme hierbei, die nicht im Super POM definierten Attribute: {\bf modelVersion} (Version der Maven-Implementierung), {\bf groupId}, {\bf artifactId} und {\bf version}. Aus diesem Grund sind diese Attribute auch die einzigen, die beim Anlegen eines neuen Projekts angegeben werden m"ussen. Hieraus ergibt sich eine Standard-Verzeichnisstruktur wie sie im Listing \ref{lst:maven:ubermep-core-dir} anhand eines Ausschnitts des Ubermep-Core Unter-Moduls zu erkennen ist.

\lstinputlisting[language=Python,label=lst:maven:ubermep-core-dir]{lst/ubermep-core-dir.txt}

Im Hauptverzeichis befindet sich in der {\tt pom.xml} das POM. Im Unterverzeichnis {\tt src/main/java} befindet sich der Quellcode. Die Quellcode-Dateien werden dann in Verzeichnissen entsprechend der Namen der packages abgelegt. Zus"atzlich ben"otigte Ressourcen finden sich im Verzeichnis {\tt src/main/resources}. Das Verzeichnis {\tt target/classes} enth"alt schlie"slich die compilierten Quellcode-Da\-teien, sowie das gesamte compilierte gepackte Projekt. Das Ausf"uhren einzelner Build-Phasen, sogenannter \emph{Goals}, erfolgt dabei im Wurzelverzeichnis des Projekts. Im Falle von \emph{compile} erfolgt dies durch das Ausf"uhren des Befehls \texttt{mvn compile}. 

\subsection{Maven in dieser Arbeit}
Diese Arbeit verwendet Maven um die verschiedenen teils voneinander abh"angigen Module zu trennen. Hierbei gibt es eine {\bf Parent POM}: das {\bf Ubermep-Parent}. Diese Parent-POM ist zust"andig f"ur s"amtliche Untermodule wie {\bf Uber\-lay-Parent} oder {\bf Ubermep-Core}.

\section{Mockito}
\label{sec:mockito}

\nomenclature{TDD}{Test Driven Development}
\nomenclature{BDD}{Behavior Driven Development}

Mockito \cite{mockito} ist eine Java-Programmbibliothek zum Erstellen von sogenannten {\it Mock}-Objekten in Unit-Tests. {\it Mock}-Objekte dienen dabei als Platzhalter f"ur tats"achliche Java-Objekte. Diese Mock-Objekte helfen dann dem Entwickler dabei das korrekte Verhalten von Interfaces zu testen. Der Vorteil hierbei: die zu testenden Interfaces k"onnen isoliert von ihren Abh"angigkeiten, also ihrer Umgebung getestet werden. Diese {\it Mock}-Objekte k"onnen dabei {\it Testgetriebene Entwicklung} (TDD) bzw. {\it Verhaltensgetriebene Softwareentwicklung} (BDD) vereinfachen.

\subsection{Mockito in dieser Arbeit}
In dieser Arbeit wird Mockito daf"ur eingesetzt, um zu "uberpr"ufen und sicherzustellen, dass ausgew"ahlte Vorg"ange das tun, wof"ur sie vorgesehen sind. Dies wird sowohl "uber die korrekten R"uckgabewerte als auch das "Uberpr"ufen interner Aufrufe gew"ahrleistet. Das folgende vereinfachte Beispiel in den Listings \ref{lst:phonebook} und \ref{lst:mockito} soll dabei die hier verwendete Funktionsweise veranschaulichen:

\lstset{tabsize=4,language=Java}
\begin{lstlisting}[label=lst:phonebook,caption=Klasse PhoneBook]
public class PhoneBook{
	private final List<PhoneBookEntry> entryList;

	public PhoneBook(List<PhoneBookEntry> entryList) {
		this.entryList = entryList;
	}

	public void add(PhoneBookEntry entry){
		this.entryList.add(entry);
	}
}
\end{lstlisting}
Es wird angenommen, es existiert das Interface {\it PhoneBookEntry}. Jetzt kann das korrekte Verhalten der Methode {\it add(PhoneBookEntry entry)} der Klasse {\it PhoneBook} wie folgt getestet werden:
\lstset{tabsize=4,language=Java,emph={verify},emphstyle=\it,literate={�}{{\"a}}1}
\begin{lstlisting}[label=lst:mockito, caption=Testklasse PhoneBookTest]
@RunWith(MockitoJUnitRunner.class)
public class PhoneBookTest {
	@Mock
	List<PhoneBookEntry> entryList;

	@Mock
	PhoneBookEntry entry;

	PhoneBook phoneBook; //zu testendes Objekt
	
	@Test
	public void testAdd(){
		phoneBook = new PhoneBook(entryList);

		phoneBook.add(entry);
		verify(entryList, times(1)).add(entry);
							// Best�tigen das intern die
							// Methode entryList.add(entry)
							// einmal aufgerufen wird
	}
}
\end{lstlisting}
Der Vorteil hierbei: {\it nur} das Verhalten der Klasse {\it PhoneBook} wird getestet, nicht das der Interfaces {\it PhoneBookEntry} bzw. {\it List}.

\subsection{Weitere Funktionen}
An dieser Stelle soll auf die weiteren vielf"altigen Funktionalit"aten von Mockito nicht weiter eingegangen werde. Es sei hier nur auf die Dokumentationsseite von Mockito \cite{mockito_doc} verwiesen, wo diese n"aher beschrieben und anhand von Beispielen leicht nachzuvollziehen sind.
