\chapter{Zusammenfassung und Ausblick}
\label{cha:zusammenfassung}

Im Rahmen dieser Bachelorarbeit wurde eine Middleware-L"osung f"ur den Aufbau eines Peer-to-Peer-basierten Overlaynetzwerks entwickelt. Die entworfene Middleware, namens \emph{Ubermep}, erlaubt Peers den Nachrichtenaustausch anhand von \emph{Message Exchange Pattern}.

Dazu wurden zu Beginn der Arbeit die ben"otigten Message Exchange Pattern: \emph{Unicast}, \emph{Multicast}, \emph{Single Request Single Response}, \emph{Single Request Multi Response}, \emph{Multi Request Multi Response} sowie \emph{RPC} identifiziert. In der Entwurfsphase wurde dann ein Protokoll entwickelt welches die verschiedenen Message Exchange Pattern abbildet. Anschlie"send wurde die Architektur der Middleware auf Basis des Frameworks JBoss Netty, einem Framework zur Erstellung von netzwerkbasierten Applikationen, und Uberlay, einem hochperformanten Overlay-Netzwerk f"ur kleine Netzwerke, entworfen. Im Anschluss wurde das Protokoll sowie die Java-Bibliothek anhand der konzipierten Architektur implementiert und getestet. Abschlie"send wurde die Implementierung auf Basis verschiedener Performanzkriterien evaluiert.

Die Evaluation der Implementierung zeigte dass die Anforderungen an gute Performanz und niedrige Latenzen erf"ullt sind. Dar"uber hinaus gibt es nur einen geringen zus"atzlichen Overhead gegen"uber dem Nachrichtenaustausch ohne Message Exchange Patterns bei direkter Verwendung von Uberlay.

Das Ergebnis der Arbeit liefert daher eine performante Middleware-L"osung, welche Entwickler verteilter Anwendungen eine Java-Bibliothek bietet, die Nachrichten der oben genannten Muster in einem Overlay-Netzwerk versendet.

Die Einsatzgebiete von Ubermep, wie in der Aufgabenstellung definiert, sind vielf"altig und reichen von der Verwendung f"ur den Nachrichtenaustausch kleinerer Datenmengen in der Infrastruktursoftware \emph{Testbed-Runtime} bis zur Verwendung f"ur den Nachrichtenaustausch mittlerer bis gro"ser Datenmengen in File-Sharing-Netzwerken. Die Unterst"utzung entfernter Funktionsaufrufe mittels RPC, erm"oglicht des weiteren die Nutzung verteilter Rechenresourcen z.B. im Bereich des Grid-Computing.

F"ur die Zukunft ist geplant das Publish-Subscribe Pattern f"ur verteilte Systeme, in Ubermep zu integrieren. 
Betrachtet man die Peers im Overlay-Netzwerk als Produzenten und Konsumenten von Nachrichten so senden in der derzeitigen Implementierung von Ubermep die Produzenten die Nachrichten direkt an die Konsumenten. Im Publish-Subscribe Pattern werden diese voneinander entkoppelt. Ein Sender (sogenannter Publisher) sendet eine Nachricht an einen Vermittler (Broker). Empf"anger (sogenannte Subscriber) bekunden Interesse an ausgew"ahlten Informationen, z.B. bestimmt "uber Filter auf sogenannte \emph{Topics}. Der Broker "ubernimmt dann den Transport der Nachricht zu den registrierten (also interessierten) Subscribern. Die Realisierung in Ubermep k"onnte dabei "uber die bereits exisitierenden Nachrichtentypen der Message Exchange Pattern erfolgen. Zus"atzlich w"urde aber die Funktionalit"at des Brokers ben"otigt werden.