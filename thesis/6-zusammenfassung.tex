\chapter{Zusammenfassung und Ausblick}
\label{cha:zusammenfassung}

Im Rahmen dieser Bachelorarbeit wurde eine Java-Bibliothek zum Aufbau eines Peer-to-Peer Overlay Netzwerks entwickelt, welches den Austausch von Nachrichten erm"oglicht.

Dazu wurden zu Beginn des Projekts Anforderungen spezifiziert. Aus der grundlegendsten Funktionalit"at, dem \emph{Reliable Messaging} wurde schlie"slich in der Entwurfsphase ein \emph{Message Exchange Pattern} definiert und spezifiziert, welches mittels einem daraus resultierenden Architekturkonzept umgesetzt und implementiert wurde. Diese Umsetzung wurde schlie"slich auf Basis der Anforderung der \emph{Effizienz des Nachrichtenaustauschs} getestet.

Die Evaluation der Implementierung ergibt zum Einen, ein akzeptables Verhalten der "Ubertragungszeit der Nachrichten in Abh"angigkeit der Nachrichtengr"o"se. Zum Anderen ergibt sie keinen zus"atzlichen Overhead zum vergleichbaren Nachrichtenaustausch mittels dem zugrundeliegenden Uberlay-Projekt. Der sich ergebende Overhead der Request-Response Pattern im Vergleich zu Nachrichtentypen des One-way Pattern ergibt sich durch die zus"atzliche Wartezeit auf die Response und bewegt sich daher im Rahmen des erwarteten. 

Da diese Implementierung hochgradig vom zugrundeliegenden Uberlay-Projekt abh"angig ist, liegt f"ur die Zukunft prim"ar das Augenmerk auf der Optimierung von Uberlay. Optimierungsans"atze liegen hier vor allem in Bezug auf eine Reduktion der "Ubertragungszeit des Nachrichtenversands von Uberlay, sowie eine Verbesserung der Performanz  f"ur den Aufbau von Overlay-Netzwerken, in Bezug auf sich hoch dynamisch "andernde Netzwerke. Da bislang Uberlay nur eine statische Festlegung der Aktualisierungszeit bietet, k"onnte hier, z.B. durch die dynamische Anpassung der Aktualisierungszeit der Routing-Tabellen, eine deutlich bessere Performanz erreicht werden. Allerdings sollte man sich immer der Problematik des gegebenfalls daraus resultierenden zus"atzlich aufkommenden Netzwerkverkehrs bewu"st sein.

Die Einsatzgebiete von Ubermep sind vielf"altig und reichen von der Verwendung f"ur den Nachrichtenaustausch kleinerer Datenmengen in Sensornetzwerken bis zur Verwendung f"ur den Nachrichtenaustausch mittlerer bis gro"ser Datenmengen in File-Sharing-Netzwerken. Die Unterst"utzung verteilter Funktionsaufrufe mittels RPC, erm"oglicht des weiteren die Nutzung verteilter Rechenresourcen z.B. im Bereich des Grid-Computing. Auch die Funktionalit"at von WebServices, beispielsweise durch verteilte Bereitstellung von web-basierten Diensten auf Peers, ist ein m"oglicher Anwendungsbereich.