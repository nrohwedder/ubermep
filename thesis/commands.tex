%------------------------------------------------------------------------------
%- PAKETE
%------------------------------------------------------------------------------
	%DIN A4
		\usepackage{a4}   

	%Fancy headers
		\usepackage{fancyhdr}
	
	%Sprache einstellen (Inhaltsverzeichnis, ...)
		\usepackage[english, ngerman]{babel} %american,italian,german
	
	%Euro Zeichen
		\usepackage{eurosym}	
		
		\usepackage[bookmarks=true,
					bookmarksopen=true,
   					% Lesezeichen ausgeklappt
					bookmarksnumbered=true,
					colorlinks=true,
				   	%Einf�rbung von Links
					linkcolor=black
					% Linkfarbe: schwarz					
					]
				    % Anzeige der Kapitelzahlen am Anfang der Namen der Lesezeichen
				   {hyperref}
		
	% Vereinfachtes Eingeben von Leerschl�gen hinter Shortcut-Commands
	% Beispiel: \newcommand{\DNA}{desoxyribose nucleid acid\xspace}
		\usepackage{xspace}
	
	%Sortierte Literaturverweise
		\usepackage{cite}
			
	%Grafiken
		\usepackage{float} %Float-Handling mit Schalter H (gleiche Position wie im Skript)
		%\usepackage{flafter} %Verhindert Figuren vor ihrer ersten Referenz
		\usepackage{placeins} %Barriere f�r Float-Umgebungen erzeugen mit \FloatBarrier
	
	%Verbessertes Beschriften mir div. Optionen
		\usepackage{caption}
	
	%Zusaetzliche Symbole und Schriften (ams: american mathematical soc)
		\usepackage{amssymb}
		%\usepackage{amstext}
		%\usepackage{amsfonts}
		%\usepackage{amsbsy}
		%\usepackage{amscd}
		%\usepackage{latexsym}

	%Text Companion fonts which provide many text symbols (such as baht, bullet, copyright, musicalnote, onequarter, section, and yen) in the TS1 encoding.
		\usepackage{textcomp}
	
	%Drehen von Text, Tabellen, Seiten
		%\usepackage{rotating}
	
	%including graphics files, rotating parts of a page, and scaling parts of a page
		\usepackage{graphicx}
	
	%Nice drawing package
		%\usepackage{tikz}               
	
	%besserer eps import: \eps import ERSETZEN durch \epsfig
		\usepackage{epsf}
	
	%Farbunterst�tzung (ausserhalb der Bilder)
		\usepackage{color}
	
	%Postcript einbinden, wobei Text ersetzt werden kann
		%\usepackage{psfrag}
	
	%F�r den Index
		\usepackage{makeidx}
		\makeindex %Muss vor begin{document}, sonst passiert nix
	
	%Erleichterungen f�rs Deutsche inkl Silbentrennung
		%\usepackage[german]
	
	%Ensure minimal spacing of table cells (http://www.ctan.org/tex-archive/help/Catalogue/entries/cellspace.html)
		\usepackage{cellspace}
	
	%Direkte Eingabe von Umlauten mit Angabe von Schriftsatz
	%in Kombination mit 'german' sind jetzt � direkt erlaubt!
		\usepackage[latin1]{inputenc}
		%\usepackage{ngerman}
	
	%Source code Listings 
		\usepackage{listings}
	
	%Darstellung von Algorithmen
		\usepackage{algorithm}
		\usepackage{algorithmic}
	
	%Subfigures
		\usepackage{subfigure}

%-- Fieser Hack f�r Subfigures (braucht man, um lstlistings im Subfigures zu nutzen)
\newbox\subfigbox
\makeatletter
\newenvironment{subfloat}
{\def\caption##1{\gdef\subcapsave{\relax##1}}%
\let\subcapsave\@empty
\setbox\subfigbox\hbox
\bgroup}
{\egroup
\subfigure[\subcapsave]{\box\subfigbox}}
\makeatother
	
	%Automatically adds the bibliography and/or the index and/or the contents, etc., to the Table of Contents listing.
		%\usepackage[nottoc]{tocbibind} %,notlot,notlof
	
	%St Mary Road symbols for theoretical computer science.
		%\usepackage{stmaryrd}
	
	%URL Darstellung
		\usepackage{url}

	%PDF und Standard Latex Unterscheidung
		\usepackage{ifpdf} 

	%Fancy verbose environments
		\usepackage{fancyvrb}	

	%Abk�rzungsverzeichnis
		\usepackage{nomencl}
		  \let\abbrev\nomenclature
		  \renewcommand{\nomname}{Abk�rzungsverzeichnis}
		  \setlength{\nomlabelwidth}{.25\hsize}
		  \renewcommand{\nomlabel}[1]{#1 \dotfill}
		  \setlength{\nomitemsep}{-\parsep}
		  \makenomenclature

	  \newcommand{\abk}[2]{#1\abbrev{#1}{#2}}
				
	%With \usepackage{ulem}, you have the following new commands:
		%    * \uline{important} underlined text
		%    * \uuline{urgent} double-underlined text
		%    * \uwave{boat} wavy underline
		%    * \sout{wrong} line drawn through word
		%    * \xout{removed} marked over with //////.
		%    * {\em phasized\/} and \emph{asized} In LaTeX, by default, these are underlined; use \normalem or [normalem] to restore italics
		%    * \useunder{\uwave}{\bfseries}{\textbf} use wavy underline in place of bold face 
		%Note that this package changes \em and \emph to be underline. To change this behavior back to normal, use the \normalem command, for example
		%\usepackage{ulem}
		%\normalem
		%\usepackage[normalem]{ulem}

	  %\newcommand{\markup}[1]{\uline{#1}}	

	% package to customize the three basic lists (enumerate, itemize and description) 
	% by means of a set of parameters, and to clone them to define new "logical" lists.
		\usepackage{enumitem}
		\setitemize{enumsep=-3pt}
		\setitemize{itemsep=-3pt}

	%Definitionen
		\usepackage{theorem}
		\newcounter{theorem}
		\newtheorem{definition}[theorem]{Definition}

	%Zitate
		\newcounter{quotectr}
		\newtheorem{myquote}[quotectr]{Zitat}

%------------------------------------------------------------------------------
%- Layout
%------------------------------------------------------------------------------

	%Tiefe des Inhaltsverzeichnisses und der Nummerierung der Kapitel
		\setcounter{secnumdepth}{2}
		\setcounter{tocdepth}{2}

	%Call this after each chapter to avoid headlines on empty pages
		\newcommand{\chapterfin}{\clearpage{\pagestyle{empty}\cleardoublepage}}
		\newcommand{\sectionfin}{\clearpage{\pagestyle{empty}\cleardoublepage}}

	% Listings schoen machen 
		\renewcommand*\ttdefault{txtt}
	
		\lstset{%
		  breaklines=true,
		  basicstyle=\ttfamily\footnotesize,%
		  moredelim=[is][\fontseries{lt}\bfseries]{|}{|},%
		  captionpos=b,%
		  numbers=left,%
		  tabsize=1,%
		  numberstyle=\tiny,%
		  numbersep=6pt,%
		  frame=lr,%
		  framesep=0pt,%
		  framexleftmargin=5pt,%
		  framextopmargin=0pt,%
		  framexbottommargin=0pt,%
		  xleftmargin=15pt,%
		  xrightmargin=15pt,%
			abovecaptionskip=0pt,%
			belowcaptionskip=-0pt,%
		}	
		
		\lstdefinelanguage{XMLSchema}
			{morekeywords={schema,element,annotation,appinfo,complexType,simpleType,choice,all,sequence},		
			sensitive=true,
%			morecomment=[l]{//},
%			morecomment=[s]{/*}{*/},
			morestring=[b]",
		}
		
		\lstdefinelanguage{ASN1}
			{morekeywords={},		
			sensitive=true,
%			morecomment=[l]{//},
%			morecomment=[s]{/*}{*/},
			morestring=[b]",
		}
		
	
	% Font
	%
	%	Danach muss man die Standardschriftart setzen mit dem Befehl \fontfamily{abr}\selectfont, 
	% der f�r das gesamte restliche Dokument gilt, oder mit {\fontfamily{abr}\selectfont Some Text} 
	% um nur den eingeklammerten Bereich zu betreffen. abr ist die Abk�rzung f�r die Schriftart. Die 
	% h�ufigsten sind ptm (Times), phv (Helvetica), pcr (Courier), pbk (Bookman), pag (Avant Garde), 
	% ppl (Palatino), bch (Charter), pnc (New Century Schoolbook), pzc (Zapf Chancery), put (Utopia ).

	% Sch�nerer tt font:
		%\renewcommand{\ttdefault}{pcr}
		%\selectfont
	
	% Times
		%\usepackage{times}
		
	%	Helvetica
			%\usepackage{helvet}
			%\renewcommand{\familydefault}{\sfdefault}
			%\renewcommand{\familydefault}{phv}
			%\fontfamily{abr}\selectfont
			
	%	Courier
			%\usepackage{courier} \raggedright
			%\renewcommand{\familydefault}{\ttdefault}
	
	% Absatzformatierungen:
	% Keeps the distance between paragraphs constant
		\setlength{\parskip}{1.5ex plus 0.0ex minus 0.0ex}
		\setlength{\parindent}{0pt}
	
	% Modify the placement of figures: from faq source: You can adjust the cut-off value if you like, 
	% but it makes no sense to go higher than .95 (LaTeX's default value is only .5). Also, the first 
	% 3 values should be equal, and the last should be 1 - \floatpagefraction.  Otherwise, you are 
	% likely to get floats flushed to the end. 
		\renewcommand{\floatpagefraction}{0.9}
		\renewcommand{\topfraction}{0.9}
		\renewcommand{\bottomfraction}{0.9}
		\renewcommand{\textfraction}{0.1}
		\renewcommand{\textfloatsep}{5mm}
	
	% Zeilenabstand
		\renewcommand{\baselinestretch}{1.0}

	% Fancyheaders 	
		\fancyhf{} % Delete all fields
		%\fancyhead[EL,OR]{\thepage}
		\fancyhead[EL]{\nouppercase{\leftmark}}
		\fancyhead[OR]{\nouppercase{\rightmark}}
		\fancyfoot[EL,OR]{\thepage}	
		
	% Itemize look and feel
		\renewcommand{\labelitemi}{\rule[+0.9mm]{2.7pt}{2.7pt}}
		\renewcommand{\labelitemii}{--}
		%\renewcommand{\labelitemiii}{}
		%\renewcommand{\labelitemiv}{\#}	

	% Floats richtig benennen:
		%\floatname{algorithm}{Algorithm}
		%\renewcommand{\listalgorithmname}{Algorithmen}
		
%------------------------------------------------------------------------------
%- Textbausteine
%------------------------------------------------------------------------------

	%Helpers
		\newcommand{\todo}[1]				{{\em [#1]}\marginpar{{\bf [!!!]}} }
		
		\newcommand{\eigenname}[1]	{{\em #1}}
		
	%Deutsch
		\newcommand{\figref}[1]{Abbildung~\ref{fig:#1}}
		\newcommand{\tabref}[1]{Tabelle~\ref{tab:#1}}
		\newcommand{\equref}[1]{Gleichung~\ref{equ:#1}}
		\newcommand{\chapref}[1]{Kapitel~\ref{cha:#1}}
		\newcommand{\appref}[1]{Anhang~\ref{cha:#1}}
		\newcommand{\secref}[1]{Abschnitt~\ref{sec:#1}}
		\newcommand{\lstref}[1]{Listing~\ref{lst:#1}}
		\newcommand{\algref}[1]{Algorithmus~\ref{alg:#1}}
		\newcommand{\ssecref}[1]{Unterabschnitt~\ref{ssec:#1}}
		\newcommand{\quoteref}[1]{Zitat~\ref{quote:#1}}

	%Englisch	
		%\newcommand{\figref}[1]		{Figure~\ref{fig:#1}}
		%\newcommand{\tabref}[1]		{Table~\ref{tab:#1}}
		%\newcommand{\equref}[1]		{Equation~\ref{equ:#1}}
		%\newcommand{\algref}[1]		{Algorithm~\ref{alg:#1}}
		%\newcommand{\defref}[1]		{Definition~\ref{def:#1}}
		%\newcommand{\quoteref}[1]	{Quote~\ref{quote:#1}}
		
		%\newcommand{\chapref}[1]	{Chapter~\ref{cha:#1}}
		%\newcommand{\appref}[1]		{Appendix~\ref{cha:#1}}
		%\newcommand{\secref}[1]		{Section~\ref{sec:#1}}
		%\newcommand{\ssecref}[1]	{Section~\ref{ssec:#1}}
		%\newcommand{\sssecref}[1]	{Section~\ref{sssec:#1}}
		
	% REDEFINE UGLY STUFF
		\renewcommand{\leq}		{\leqslant}
		\renewcommand{\geq}		{\geqslant}
		\renewcommand{\epsilon}	{\varepsilon}
		\newcommand{\musec}		{$\mu sec$\xspace}
		\newcommand{\muW}		{$\mu W$\xspace}
		\newcommand{\plusminus}	{$\pm $\xspace}
	
%------------------------------------------------------------------------------
%- Worttrennung
%------------------------------------------------------------------------------
	
	%\hyphenation{Ge-samt-ozon}	
	\hyphenation{name-space}	
	\hyphenation{name-spaces}
	
	\hyphenation{ge-samten}
	\hyphenation{Ziel-addresse}
	\hyphenation{Single-Request-Single-Response}
	\hyphenation{Single-Request-Multi-Response}
	\hyphenation{Multi-Request-Multi-Response}
	%\hyphenation{}		
%------------------------------------------------------------------------------
%- Grafiken
%------------------------------------------------------------------------------

	\ifpdf
	  \DeclareGraphicsExtensions{.jpg,.pdf,.png}   % for pdftex driver
	\else
	  \DeclareGraphicsExtensions{.eps}             % for dvips driver
	\fi
	
	% Vereinfacht die Einbettung von Grafiken
	% Beispiel: \myfig[5cm]{psdatei}{�bersicht �ber das Gesamtsystem}
	\newcommand{\myfig}[3][\columnwidth]
	{
	 \begin{figure}[htbp]
		 \begin{center}
			 \includegraphics[width=#1]{img/#2}
			 \caption{#3}
			 \label{fig:#2}
		 \end{center}
	 \end{figure}
	}
	
	\newcommand{\myfigtwo}[4][\columnwidth]
	{
		 \begin{figure}[htbp]
				\begin{center}
				  \mbox
				  {
				    \subfigure[#2] 
				    { \includegraphics[width=.45\columnwidth]{img/#1-a} \label{fig:#1-a} } 
				    \quad
				    \subfigure[#3]
				    { \includegraphics[width=.45\columnwidth]{img/#1-b} \label{fig:#1-b} }
			    }
				  \caption{#4}
					\label{fig:#1}
				\end{center}
			\end{figure}
	}
	
	\newcommand{\myfigthree}[5][\columnwidth]
	{
		 \begin{figure}[htbp]
				\begin{center}
				  \mbox{
				    \subfigure[#2]
				    {
				    	\includegraphics[width=.3\columnwidth]{img/#1-a}
				    	\label{fig:#1-a}
				    } 
				    \subfigure[#3]
				    {
							\includegraphics[width=.3\columnwidth]{img/#1-b}
				    	\label{fig:#1-b}
				    }
				    \subfigure[#4]
				    {
							\includegraphics[width=.3\columnwidth]{img/#1-c}
				    	\label{fig:#1-c}
				    }
			    }	
				  \caption{#5}
					\label{fig:#1}
				\end{center}
			\end{figure}
	}
	
	\newcommand{\myfigfour}[6][\columnwidth]
	{
		 \begin{figure}[htbp]
				\begin{center}
				  \mbox
				  {
				    \subfigure[#2] 
				    { \includegraphics[width=.45\columnwidth]{img/#1-a} \label{fig:#1-a} } 
				    \quad
				    \subfigure[#3]
				    { \includegraphics[width=.45\columnwidth]{img/#1-b} \label{fig:#1-b} }
			    }
				  \mbox
				  {
				    \subfigure[#4] 
				    { \includegraphics[width=.45\columnwidth]{img/#1-c} \label{fig:#1-c} } 
				    \quad
				    \subfigure[#5]
				    { \includegraphics[width=.45\columnwidth]{img/#1-d} \label{fig:#1-d} }
			    }
			    
				  \caption{#6}
				\label{fig:#1}
				\end{center}
			\end{figure}
	}
	
